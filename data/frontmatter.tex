% TITLE PAGE
\begin{titlepage}
\centering

\vspace*{\stretch{0.5}}
The Pennsylvania State University\\The Graduate School
\vspace{\stretch{1}}

\begin{doublespace} 
{\hypersetup{urlcolor=black} {\large\bfseries \href{https://nmorse.com/dissertation}{AMERICAN POLITICS IN PERSPECTIVE: \\THE LIMITATIONS OF STATIC CONSTITUTIONS AND STATIC CONTENT}}}
\vspace{\stretch{1}}

A Dissertation in\\Political Science\\by\\Nathan Morse
\end{doublespace}
\vspace{\stretch{1}}

© 2024 Nathan Morse
\vspace{\stretch{1}}

Submitted in Partial Fulfillment\\of the Requirements\\for the Degree of
\vspace{\stretch{0.5}}

Doctor of Philosophy
\vspace{\stretch{0.5}}

May 2024
\vspace{\stretch{1}}

\end{titlepage}


% COMMITTEE
\newpage
\begin{flushleft}
The dissertation of Nathan Morse was reviewed and approved by the following:
\vspace{2em}

Christopher Zorn\\
Liberal Arts Professor of Political Science and Sociology\\
Dissertation Advisor\\
Chair of Committee

\vspace{1em} % Adds extra space

Ray Block Jr.\\
Associate Professor of Political Science and African American Studies

\vspace{1em} % Adds extra space

Suzanna Linn\\
Liberal Arts Professor of Political Science

\vspace{1em} % Adds extra space

Christopher Witko\\
Professor of Public Policy and Political Science

\vspace{1em} % Adds extra space

Michael J. Nelson\\
Director of Graduate Studies, Department of Political Science\\
Professor of Political Science

\end{flushleft}

% ABSTRACT
\newpage
\begin{center}
\textbf{Abstract}
\end{center}

% Adjust the spacing as needed for your specific abstract
Why has the United States become one of the most polarized and unequal countries in the democratic world? I argue that the American constitutional model is no longer compatible with American society. Large, diverse republics generally need more flexible institutions that are geared toward consensus building rather than majority rule to remain free and stable. Over the last half century, the American electorate has become more diverse than its institutions can handle. Americans are now gasping for a multiparty system and other updates that are not viable under the current framework. The equal representation of states in the Senate, majoritarian elections, and outdated amendment process have enabled economic inequality and polarization to rise by blocking routine maintenance to the nation’s democratic institutions and economic strategies. American democracy is now struggling not in spite of the Constitution, but because of it. A constitutional convention may be necessary to address these challenges in the long run.

Adding to the difficulty of constitutional reform, political misinformation has been getting more sophisticated—misleading charts often go viral, reaching millions of people—and paywalled PDFs are no match for modern media. Embracing dynamic data visualizations, videos, and interactive articles would help researchers advocate for policies that could strengthen American democracy. To illustrate this point, each chapter of this dissertation features an interactive data app showing how political institutions affect a variety of outcomes. I also shared animated charts adapted from these figures on social media and reflected on the experience. The dissertation as a whole is designed to inform the public and contribute to the academic literature at the same time.


